% 编码格式: "TeX:UTF-8"
% tex编译链 = xelatex
% Created by tinoryj on 2017/2/27.
% Copyright © 2017年 tinoryj. All rights reserved.
% 模版支持三级标题,如果需要插入第四级
% lipsum[1] 为模版文字,使用时自行删除
% 基于2011版格式要求设计,针对现行电子版不要求前两页信息进行设计。

%\documentclass[bwprint]{cumcmthesis}
\documentclass[withoutpreface,bwprint]{cumcmthesis} %去掉封面与编号页

\title{模版}
\tihao{A}
\baominghao{201723003002}
\schoolname{电子科技大学}
\membera{卫佳杰}
\memberb{谢沁余}
\memberc{任彦璟}
\supervisor{覃思义}
\yearinput{2017}
\monthinput{09}
\dayinput{17}

\begin{document}
 \maketitle
 
 \begin{abstract}
 %摘要正文

 
\keywords{关键字1}

\end{abstract}


\section{问题重述}
\section{问题分析}
\section{模型假设}
\begin{itemize}
	\item ...
\end{itemize}
\section{符号说明}
\subsection{名词解释}
\subsection{变量说明}
\begin{table}[!h]
\centering
\begin{tabular}{ccc}
\toprule
\makebox[0.2\textwidth][c]{符号}	&  \makebox[0.4\textwidth][c]{意义} &  \makebox[0.2\textwidth][c]{单位} \\
\midrule
$COR$ & CT旋转中心 & (坐标)\\
\bottomrule 
\end{tabular}
\end{table}

\section{模型建立与求解}
\subsection{CT系统的参数标定}
\subsubsection{模型建立}

\begin{figure}[h]
\small
\centering
\includegraphics[width=\textwidth]{q1-table2-sejie.png}
\caption{附件二探测器接收信息色阶图} \label{fig:q1-table2-sejie}
\end{figure}
\begin{figure}[h]
\small
\centering
\includegraphics[width=\textwidth]{q1-zuobiaoxi.png}
\caption{坐标系建立及旋转中心定位原理示意图(单位:mm)} \label{fig:q1-zuobiaoxi}
\end{figure}
\subsubsection{模型求解}




\section{模型评价与改进}
\subsection{模型优点}
\subsection{模型缺点}
\subsection{模型改进}



\begin{thebibliography}{9}
 \bibitem{bib:one} ....
 \bibitem{bib:two} ....
\end{thebibliography}

\section{附件清单}

\newpage
\section{附件}
\subsection{关键数据}
\subsection{程序源代码}
cpp\textcolor[rgb]{0.98,0.00,0.00}{\textbf{Input C++ source:}}
% 代码插入,请将代码文件放入code文件夹,支持语言的语法高亮。支持语言:C,C++,Java,Matlab,Mathematica,python,R,可在cls文件中自行添加。
\lstinputlisting[language=C++]{./code/a.cpp}


\end{document} 